\documentclass{article}

\usepackage{arxiv}

\usepackage[utf8]{inputenc} % allow utf-8 input
\usepackage[T1]{fontenc}    % use 8-bit T1 fonts
\usepackage{hyperref}       % hyperlinks
\usepackage{url}            % simple URL typesetting
\usepackage{booktabs}       % professional-quality tables
\usepackage{amsfonts}       % blackboard math symbols
\usepackage{nicefrac}       % compact symbols for 1/2, etc.
\usepackage{microtype}      % microtypography
\usepackage{graphicx}
\usepackage{natbib}
\usepackage{doi}
\usepackage{orcidlink}
\usepackage{natbib}

\title{The Philosophy Crisis of Theoretical Physics}

\date{January 4, 2026}

\author{ \hspace{1mm}Xinyu Yang \orcidlink{0009-0007-2600-0948}\thanks{Contact author: \href{mailto:Pana.Yang@hotmail.com}{Pana.Yang@hotmail.com} ORCID: \href{https://orcid.org/0009-0007-2600-0948}{0009-0007-2600-0948}
} \\
Independent Researcher \\
Jinan Zhensheng Middle School, Jinan 250100, Shandong, China \\
\\
doi: \href{https://www.doi.org/10.5281/zenodo.18144962}{https://www.doi.org/10.5281/zenodo.18144962}
}

% Header
\renewcommand{\headeright}{Zenodo Preprint}
\renewcommand{\undertitle}{Zenodo Preprint}

% Add PDF metadata
\hypersetup{
pdftitle={The Philosophy Crisis of Theoretical Physics},
pdfsubject={hep-th, quant-ph},
pdfauthor={Xinyu Yang},
pdfkeywords={Philosophy of Science, Natural Science, Scientific Method},
}

\begin{document}
\maketitle

\begin{abstract}
Theoretical Physics emerges from natural science and are always part of natural philosophy. Therefore, there is necessitates for us to look back at the fundamental assumptions, models and concepts of the theories already there and the philosophical underpinnings or underlying conceptual framework behind them. In this paper, we examine some of the most well-known theories and some of the most profound inconsistencies inherent in their bases. From this, we contend that the fundamental axiomatic mandate and objective of theoretical physics need to be reassessed and restructured by the humanity as a whole. This requires the physics community and the broader humanity to rethink about the epistemic structure of physics and thus could not be determined by any single person or entities. So, we provided some tentatively clues about the possible redefinition of these philosophy concepts and the possible direction to a new unified theory. Our work offers a novel perspective on the traditional theory of our time and opens up a new space for future subsequent research and discussions. 
\end{abstract}

\keywords{Philosophy of Science \and Natural Science \and Scientific Method}

\section{Introduction}

Theoretical physics has traditionally treated developing new sets of mathematical tools and novel frameworks to describe and predict nature as it's fundamental mandate. Thus, from the contributions of Thales of Miletus and Aristotle to the contemporary era, physicists have always been continuously seeking enhanced description and prediction precision. To achieve the so called empirical success, we are necessarily relied on some of the fundamental assumptions and conceptual axioms that we made in our conceptual underlying. However, as more experiments results accumulate and more structural theoretical difficulties emerges, many physicist begin to reassess these assumptions and try to overcome them. These mandate axioms, however, now has constrained the very progress it was designed to facilitate, creating an impasse and bafflement because its inconsistency. 

Consequently, in Section \hyperref[sec:established]{\ref{sec:established}} of this paper, we re-examined some of the work of the established theories and identified the fundamental inconsistency from their bases. Subsequently, we propose several potential pathways and provide some tentatively guidelines towards a new unified theoretical framework within the constraints of the current axiomatic mandates. In the following Section \hyperref[sec:crisis]{\ref{sec:crisis}}, we delineate the three fundamental mandate axioms and analyze the inconsistency and possible conflicts behind them. Finally, this paper argues that the ontological philosophy crisis of theoretical physics cannot be resolved through singular effort, but rather requires interdisciplinary revaluation.

Also, please remind that as this is a philosophy of physics related article, our discussion will only focus on fundamental expressions and observations. The math we will use here is quite basic.

\section{Observation of the established theories}
\label{sec:established}

In this section, we will re-examine the basis of some of the very foundation of modern physical theories. The discussion will mainly focus on some of the reasons of their success in achieving enhanced precision in describing, explaining and predicting nature.

\subsection{Relativity: Inertia and the broken symmetry}

From the contribution of Newton to the contemporary era, one of the key field of interest of physicists is tenacious search for new governing laws that unify space and time, or I shall say unifies space and time under varying circumstances, especially considering different type of relative motion. Thus, we gradually evolve from simplest transition laws with near \textbf{E(3)} group symmetry transition rules to Galilean transformation, which exhibits Galilean group symmetry, then near culminating in Special Relativity with Lorentz transformation rules and Poincaré group symmetry. \cite{Jackson2024} However, the attempts to unifies gravity and acceleration into our theoretical framework, we got only partial, non-consistent, and local symmetry. \cite{Pineda2025} In general relativity, using the Equivalence principle, we essentially revert back to the fundamental assumptions in classical mechanics that the inertial mass of an object is equivalent to its gravitational mass,\cite{Newton2021} and pinpoint to the fact that inertia itself could be seen as both the causes and the results of the broken symmetries. \cite{LandauLifshitzV2_2012} \cite{LandauLifshitzV1_2007}  Observing this progressive path of evolvement of physics, it is evident that symmetry itself becomes more and more subtle as we progress, creating a widening gap between the mathematical formulism that we use and the very physical nature that it is originally originated to describe mathematically.
% foud. phy. / arxiv / landau
\subsection{Quantum Mechanics: The marking of the boundary}

Quantum Mechanics is arguably one of the most profound theories that describes the foundation of the nature. At its Philosophy basis, Quantum Mechanics utilizes two arguments. \cite{Dirac2024} The first one posits that the scale, small versus large, is only relative attribute of the nature, akin to motion, and thus using increasingly smaller scales to discuss and explain macro-scale phenomena is philosophically unsustainable. The second argument concerns about the modification of the law of causality - for the reason that science at its nature is fundamentally constrained to observables and the fact that the observation itself inherently comes with perturbation, Quantum Mechanics explicitly delineates the boundary within which we could describe, explain or predict the nature. In this manner, it supports the mathematical and philosophy coherence of itself to some extend. \cite{LandauLifshitzV3_2008} However, as one of the key mandate of theoretical physics is also to serve the progress of the productivity of humanity, we are compelled by our mission to ask more about what if we shall explore the reality of nature beyond the scope of the current theoretical boundary? And what if that entities or non-entities that are not directly observable can exert a detectable impact and be observed indirectly beyond our current theoretical or practical limit. Observing this progressive path of evolvement of physics again, it is clear that marking the boundary is the bedrock on which the current theories are constructed, just like the Planck scale and the speed of light, and to move forward and serve the humanity at best we must therefore re-examine, reassess and thus transcend these known boundaries.
% foud. phy. / arxiv / landau / dirac
\subsection{Quantum Field Theory: The success of another interpolation theory}

Quantum Field Theory is one of the most established theory in the contemporary era and is one of the central researching focus of theoretical physics. One of the primary objective of Quantum Field Theory is to unify Special Relativity and Standard Quantum Mechanics, especially to explain electromagnetic phenomena using quanta entities. \cite{BerestetskiiLifshitzV4_2015} However, in this process, we have encountered increasingly formidable barriers that block us from achieving complete theoretical coherence, including the divergence of various fundamental physics constants after applying the renormalization method. \cite{Faizal2025} And consequently, in Quantum Field Theory, we utilizes methods which to some extend are analogous to second quantization procedure method to transition to a field operator description to attempt to better describe nature and mitigate these difficulties. \cite{BerestetskiiLifshitzV4_2015} Nevertheless, numerous problems affiliated to it persist to exist and we have not got a satisfactory resolution yet. As for the problems and challenges related to that, there are already extensive discussions within the scientific community; so here we advise the reader to refer to relevant literature for more detailed accounts. \cite{Zampeli2025} Herein, what we want to propose is that Quantum Field Theory itself, is a theory that achieves enhanced precision and strength through judicious approximation, and should be viewed as fundamentally an interpolative theory - it's success is, at its basis, the success of a interpolation model that unifies the Theory of Special Relativity and Standard Quantum Machines to some extent, and provides sufficient flexibility to let us further enhance its precision through adding more subsequent layers of interpolation formalism. Observing this evolutionary trajectory of physics for once another more, it becomes evident that interpolation provides a viable mechanism for theoretical discovery, \cite{Wilson1974} instead of requiring exact innovations across all theoretical expressions at the same time. And thus, interpolation theories should be seen as a way forward, rather then treating them as a suboptimal theoretical framework. \cite{Weinberg2014a}
% landau / arxiv
\subsection{The inherent existence of the arrow of time: The evolvement law of nature}

The temporal evolution law of nature is always, subject to and at the center of, all physics research till now, based on the belief that the breakdown of time-reversal symmetry and time evolution must then result in the collapse of casualty and consequently betrays our scientific mandate. \cite{Yates2025} There are Numerous of these examples illustrating our belief in this principle, such as the time-evolution axiom of Standard Quantum Machines and the Einstein Field Equations in the theory of General Relativity, which result from the axioms of it. However, as we delve deeper into the reality of nature, we further observed that the unidirectional flow of time, is, while corresponding with our routinely experience, conflicts with our philosophical commitments to symmetry and lacks a proper and robust philosophical basis. And that is notably evident in the core philosophy debate around the Second Law of Thermodynamics. \cite{LandauLifshitzV5_2011} And hence, based on this observation, we propose that our fundamental axiom regarding time, our cognition of casualty, or more broadly, our understanding of our scientific mandate and the methodology through which we could approach it shall be re-examined and revised to better align with the reality of nature and maximize the benefit to humanity. From here, the reader could refers to Section \hyperref[sec:crisis]{\ref{sec:crisis}} for a more detailed exposition of these arguments.
% arxiv
\section{The philosophy crisis of theoretical physics}
\label{sec:crisis}

In this section, we firstly articulate the three fundamental axioms of theoretical physics, subsequently identifying to the philosophy crisis of theoretical physics that we are confronting, finally conclude by posing the question regarding the fundamental mandate and objective of theoretical physics. Our discussion will mainly focus on the philosophy related dimensions of these problems, and deliberately avoiding heavy engagement with complex mathematical frameworks or tools for clarity.

\subsection{From Galileo to Einstein: The philosophical underpinnings of physics}

From the era of Galileo to the contemporary age, while our understanding of the nature has changed profoundly, the underpinning scientific mandates of us, at least in our belief, have not. Herein, we articulate three of the mandate axioms that underpin all of our uncompromising belief and scientific practice:

\paragraph{Mandate Axiom One} The mandate of theoretical physics is to describe, explain and predict nature.

\paragraph{Mandate Axiom Two} The subject of the theoretical physics mandate is accessible or at least could be approximated with arbitrary precision and the objective shall be eternal and invariant.

\paragraph{Mandate Axiom Three} The mandate of theoretical physics shall exclude self-referential propositions or any other statements regarding its own constitution. \footnote{These exclusion is justified on the basis that such kind of propositions fail to prompt the increase of productivity, creativity or other beneficial factor for humanity.}

These three mandate axiom constitute the foundation of our scientific belief till now and serve as the bedrock of our current physics theories though the interpretations may vary. From here, We direct the reader to the subsequent subsection for more discussion.

\subsection{The Crisis: The mismatch between theoretical physics demands and the underlying principles}

In this section, we will mainly discuss the main content of the philosophy crisis of theoretical physics. This section will mainly be structured into two parts: In the first subsection, we will retrospectively examine the established theories, analyzing their interpretations of the Mandate Axioms, and pinpointing the underlying inherent mismatch between the demands of theoretical physics and its foundational principles (i.e., to describe, explain, and predict nature). In the second part, we will discuss about the philosophy inconsistency between the Mandate Axioms themselves and our mandate as an entirety, thus, ultimately leading to the questions posed in the subsequent section.

\subsubsection{The reality of nature}

For most of the humanity's history, it had been assumed that we could measure the nature with arbitrarily high precision, and only constrained our experimental capability. In Standard Quantum Mechanics, we introduce the infeasibility of complete exactness of all physics states for a single microscopical physics system, yet due to our interaction speed could in theory be about infinity (i.e, the time period needed for the collapse of the wavefunction), the precision we can get remains also arbitrary in theory. \cite{LandauLifshitzV3_2008} And in Relativity, although we have introduced another hard boundary, that is to say the speed of light, and also features the broken symmetries which might permit 'non-local' interaction if given special space-time folds (Reader could refers to relevant literature for more detailed accounts about the interesting \textbf{ER = EPR} conjecture), results here remain theoretically obtainable with arbitrary precision. \cite{LandauLifshitzV2_2012} So in these contexts, we found these two theoretical framework appear both mutually and independently coherent. Conversely, within the theoretical framework of the Quantum Field Theory, the finite speed limit of any interaction which is detectable (the speed of light in the vacuum) compounds with the infeasibility of complete exactness from Standard Quantum Mechanics, we could not get results at any arbitrary precision, and thereby directly conflict with the Mandate Axiom Two and from here stems out the different kind of inconsistencies between Relativity and Standard Quantum Mechanics. \cite{BerestetskiiLifshitzV4_2015} As these inconsistencies are already been extensively discussed in the scientific community, \cite{Weinberg1979} We advise the readers to refer to the relevant literature for more detailed accounts on these issues. \cite{Weinberg2014b}

\subsubsection{The philosophy inconsistency between the mandate axioms}

In the subsection above, we could observe that on the path toward enhanced precision, the nature of our mandate becomes increasingly more vague. So to adhere to it, we begin to develop divergent interpretations of it, such as both the theory of Relativity and the Copenhagen interpretation of Standard Quantum Mechanics is, to some extend, effectively redefines the notions of precision, eternal and invariant. \cite{Weinberg2014a} Thus, it is evident that our current comprehension of our mandate or the Mandate Axioms themselves, may be inaccurate or ill-posed. Furthermore, the Gödel's incompleteness theorems \cite{Godel1931} also reminds us that complete eternal, invariant and philosophical coherence may be unachievable under the limitation of Mandate Axiom Three. And that leads inevitably to the questions around the fundamental mandate and objective of theoretical physics, which we will address in the subsequent section.
% arxiv / landau / dirac / feymman
\subsection{The Question: The fundamental mandate and objective of theoretical physics}

Based on the preceding sections, it is increasingly obverses that our comprehension of our mandate - the fundamental mandate and objective of theoretical physics and science - or our mandates themselves, is mismatched with the reality of nature and our philosophy requirements intrinsic to the discipline. And thus this brings us to the critical question that we are facing: What should our revised understanding to our mandate and objective be, to match our increasingly demand for more enhanced theoretical precision, more unified view of nature and greater productivity of humanity?

Theoretical physics and the broader scientific endeavor, in the history of humanity, have always played and shall continue to play, a vital rule in the progress of our human civilization. Consequently, the critical question above requires the physics community and the broader humanity to rethink and reassess the epistemic structure of physics and thus precluding its determination by any single individual or entity. Accordingly, in this article, we will not propose any direct and definitive answer to this problem, but will instead only focus on discussing the possible avenues for future academic and science progress.

Thus far, the physics community has already explored several paths out of these crisis, including, but not limit to, the Effective Field Theory, the string Theory, the Lattice Quantum Chromodynamics and many other theoretical attempts. Conclusively, these attempts could be categorized as: endeavors toward enhanced interpolation, efforts to define higher new theoretical boundaries, or attempts of revisiting the lost symmetry, etc. Given the breadth of these specialized areas of theoretical physics and the lack of expertise in these specific fields among us as the authors, here the reader is advised to refer to relevant literature for more detailed accounts of these specific approaches and the consequent issues.

To provide illustrative inspirations for the path forward, here we will briefly discuss two possibilities. The first one involves utilizing functional analysis tools to construct a new interpolation-based theoretical framework which, to some extend, akin to the Standard Quantum Mechanics and the Quantum Field Theory. Such a framework might necessitate a break from the Einstein's Principle of Relativity and thereby interpreting 'eternal and invariant' as attribution of the evolution of the governing laws of physics themselves. Furthermore, it might interpret dark energy and/or dark matter as results of the broken symmetry of the evolution of those physics laws. The second one posits the possibility of a theoretical framework that interprets dark energy and/or dark matter as entity possessing a different evolution dimension other then time. However, the reader shall notice that these remains only conjectures and are intended here solely for inspirational purposes. Aberrant experimental results in medium to low energy regimes necessitate interpretations using method beyond the current interpolative or non-interpolative theories to achieve further progress. Nevertheless, to support further theoretical constructions, substantial experimental breakthroughs are required to warrant the abandonment of non-enhanced theories that we developed. Finally, the Ockham's Razor Principle shall always be seen as a paramount guiding principle of science that we shall always respect. \cite{Stromme2025} 

\section{Conclusion}

In this paper, we first identified the philosophy underpinnings of several established theories and proposed novel interpretations and observations around their internal philosophical meanings. Subsequently, we pinpointed to the mandate axioms and highlighted the critical conflicts between them and either natural reality or our intrinsic philosophical commitments. Though our work, we reveal the philosophy crisis embedded within the current theoretical framework, fostering possible avenues out of the crisis and compelling a comprehensive reassessment and restructure of the fundamental axiomatic mandate and objective of theoretical physics.

Behind the current philosophy crisis, one must always remind that theoretical physics derives from natural philosophy and always remains an integral part of natural philosophy. As is the case with many long-enduring traditional or non-traditional philosophical problems, the final truth of nature and reality may well be non-absolute or eventually exist outside the constraints of our current scientific and theoretical physics mandate. And as physicists, we shall consistently uphold our scientific mandates, practice it diligently and wisely, and dedicate all of our passion to the physical and natural realities and theoretical constructs that we cherish before and after. Regardless the final resolution of the philosophy crisis of theoretical physics is, always remembering and maintaining this commitment will necessarily empower all of us to continue progressing on the path towards truth without any apprehension.

In closing, while our work provides a novel perspective into the philosophy crisis of theoretical physics, it does not offer a direct or definite solution to it. Furthermore, in order to avoid infinite regress and logical or causality recursion, our analysis is grounded on the basis of our own observation and axioms, which necessitates critical reviews by the broader scientific community and physics community as well. Future researches and work, therefore, should determine or at least rigorously discuss the proper method of modifying our understanding of our mandate, or the causality itself, to instruct future theoretical research and discoveries.

\section{Acknowledgment}

We extend our deepest gratitude towards the Physics Olympic Coaching Team of Jinan Zhensheng Middle School for their diligent administrative support throughout the research. Also, special thanks is given to Yinxian Li for the sharing of his critical insights into Classical Mechanics and his continuous encouragement. The author is particularly grateful for the unwavering support and companionship received from Zining Li during these challenging phases of this researching project. We are particularly grateful to all those whose reviews, discussions and engagements, though not explicitly mentioned here, have informed and transformed our critical reflection, speculation and research.

\bibliographystyle{plain} 
\bibliography{references}

\end{document}
